\documentclass[conference]{IEEEtran}
\IEEEoverridecommandlockouts

\usepackage{cite}
\usepackage{amsmath,amssymb,amsfonts}
\usepackage{algorithmic}
\usepackage{graphicx}
\usepackage{textcomp}
\usepackage[utf8]{inputenc}
\usepackage[T1]{fontenc} 
\usepackage{listings}
\lstset{language=bash,
  numberstyle=\footnotesize,
  basicstyle=\footnotesize,
  numbers=left,
  stepnumber=1,
  frame=shadowbox,
  breaklines=true}
\usepackage{color}

\usepackage{tabularx,ragged2e,booktabs}
\newcolumntype{C}[1]{>{\Centering}m{#1}}
\renewcommand\tabularxcolumn[1]{C{#1}}
\usepackage[most]{tcolorbox}

\usepackage[english,ngerman,brazilian]{babel}

\def\BibTeX{{\rm B\kern-.05em{\sc i\kern-.025em b}\kern-.08em
    T\kern-.1667em\lower.7ex\hbox{E}\kern-.125emX}}
    
\begin{document}

\title{Projeto Demonstrativo 2 - Calibração de Câmeras}

\author{\IEEEauthorblockN{Frederico Guth (18/0081641)}
\IEEEauthorblockA{\textit{Tópicos em Sistemas de Computação, ,} \\
\textit{Turma TC - Visão Computacional (PPGI)}\\
\textit{Universidade de Brasília}\\
Brasília, Brasil\\
fredguth@fredguth.com}
}

\maketitle

\begin{abstract}
descrição curta do trabalho e do relatório, possivelmente indicando conclusões.
\end{abstract}


\section{Introdução}
Uma câmera é um instrumento de aquisição de imagens. Conhecendo seus parâmetros intrínsecos, como distância focal e distorção da lente, e extrínsecos, sua rotação e translação no sistema de coordenadas do mundo real, é possível estimar a posição 3D de um objeto a partir de sua imagem\cite{tese}, o que possibilita diversas aplicações: por exemplo, a mensuração da altura de pessoas registradas em vídeos de camêras de segurança ou a estimativas de posições de atletas em campo, entre outras.

\subsection{Objetivos}
Os objetivos deste projeto são a aplicação prática da teoria de calibração de câmeras e o desenvolvimento de uma "régua visual", capaz de medir um objeto através da sua imagem.

\section{Revisão Teórica}
Os objetivos deste projeto são a aplicação prática da teoria de calibração de câmeras e o desenvolvimento de uma "régua visual", capaz de medir um objeto através da sua imagem.

\subsection{Modelo de Câmera com Coordenadas Homogêneas}
O modelo de câmera estenopeica (pinhole) faz um mapeamento geométrico do mundo 3D para o plano da imagem 2D.\cite{unicamp}
imagem, f, alpha, px, py [Qut]

Se os pontos do mundo (X) e da imagem (x) são representados por coordenadas homogêneas, podemos expressar matematicamente a projeção da câmera como uma matriz\cite{tese}:

lambdax = PX,

onde lambda é um fator de escala e P é a matriz 3x4 de projeção, também chamada matriz de calibração.

Sendo X coordenadas euclidianas, P pode ser decomposto em duas entidades geométricas: os parâmetros intrísecos e extrísecos de calibração [tese]

P = K(Rt), onde t é -R . C~ (2) \cite{Hartley2004}

Os parâmetros intrísecos de calibração descrevem a transformação entre a imagem ideal e a imagem em pixels

K = (fI|po);

e os extrínsecos são a rotação e translação que transformam pontos no espaço do objeto para pontos no espaço da imagem e vice-versa. [tese]

Como há 6 graus de liberdade nos parâmetros extrínsecos e 5 nos intrísecos, é necessário pelo menos 6 correspondências {xi <-> Xi} do mesmo ponto no espaço da imagem e no espaço do objeto para obter P [tese]. 

Como há um erro inerente nas medidas experimentais, para melhorar a qualidade da estimativa é preciso usar n > 6 correspondências (como será visto na seção ..., usaremos 48). Como não há uma única matriz P que resolve esse sistema de equações é adcionar restrições.  

Um método comum é adicionar a restrição p34 = 0\cite{Hartley2004}[tese], mas essa abordagem não garante que não existam configurações em que o resultado com a restrição adicional é degenerado. Uma melhor melhor abordagem[tese] é fazer:
P = arg min....
onde d(xi ,P'Xi) é a distancia euclidiana entre o ponto observado e o estimado.

A biblioteca OpenCV usa essa última abordagem e aplica o método Levenberg-Marquant para resolver a minimização. 

\subsection{Distorções}

O modelo até aqui descrito descreve uma câmera ideal, mas as lentes das câmeras reais podem gerar distorções.  Essas distorções também são parâmetros intrínsecos que precisam ser considerados. 

A distorção radial causa uma curvatura no mapeamento de retas\cite{unicamp}.
imagem curva -> reta

A correção dessa distorção pode ser modelada da seguinte maneira: 
.. math::

    % x_{corrected} = x( 1 + k_1 r^2 + k_2 r^4 + k_3 r^6) \\
    % y_{corrected} = y( 1 + k_1 r^2 + k_2 r^4 + k_3 r^6)

Outra distorção comum é a tangencial, que ocorre quando o plano da lente não está alinhado perfeitamente em paralelo ao plano da imagem. Para corrigir:

% .. math::

%     x_{corrected} = x + [ 2p_1xy + p_2(r^2+2x^2)] \\
%     y_{corrected} = y + [ p_1(r^2+ 2y^2)+ 2p_2xy]


Esses cinco parâmetros são conhecidos como coeficientes de distorção:

% .. math::

%     Coeficientes  \; de \; distorção=(k_1 \hspace{10pt} k_2 \hspace{10pt} p_1 \hspace{10pt} p_2 \hspace{10pt} k_3)

[opencv-câmera calibration]
\section{Metodologia}
O modelo da câmera estenopeica e seus parâmetros foram descritos na seção . Nesta seção, descrevem-se como estimá-los experimentalmente.

\subsection{Materiais}
Foram utilizados o seguintes materiais:
Uma tábua de compensado de WW x HH;
Papel contact
Um padrão de calibração xadrez impresso em papel A4
Uma trena
Uma régua
Computador MacBook Pro (Retina, 13-inch, Early 2015), Processador Intel Core i5 2,7 GHz, 8GB de RAM
- Python 3.6.3 :: Anaconda custom (64-bit)
- OpenCV 3.4.0

\subsection{Mensuração de segmentos de pixels em imagens}
Desenvolveu-se um programa que simplesmente abre uma imagem jpg e captura cliques do mouse formando uma linha entre o primeiro e o segundo clique. 

Calcula-se também a distância:
% ||p2 - p1||_2

onde  p1 e p2 são vetores que representam os pontos obtidos.  A norma L2 é calculada usando a função np.linalg.norm.

% Foram utilizados o seguintes materiais:
% \begin{itemize}
% \item Computador MacBook Pro (Retina, 13-inch, Early 2015), Processador Intel Core i5 2,7 GHz, 8GB de RAM
% \item Python 3.6.3 :: Anaconda custom (64-bit)
% \item OpenCV 3.3.0
% \end{itemize}
\section{Resultados}
OA discussão visa comparar os resultados obtidos e os previstos pela teoria. Deve se justificar eventuais discrepâncias observadas. As conclusões resumem a atividade de laboratório e destacam os principais resultados e aplicações dos conceitos vistos.
\section{Discussão e Conclusões}
OA discussão visa comparar os resultados obtidos e os previstos pela teoria. Deve se justificar eventuais discrepâncias observadas. As conclusões resumem a atividade de laboratório e destacam os principais resultados e aplicações dos conceitos vistos.
% \begin{itemize}
% \item Poderíamos utilizar uma abordagem multiprocessada e mover o processamento da coloração para uma thread separada da leitura dos frames do vídeo.
% \item Poderíamos usar a OpenCV para converter os frames de vídeo para o espaço de cores HSL e checar a similaridade dos pixels nesse outro espaço.
% \end{itemize}
Nenhuma dessas ideias, entretanto, faziam parte do escopo do projeto e ficam como sugestão para novas pesquisas.

\selectlanguage{brazilian}
\bibliographystyle{IEEEtran}
\bibliography{references}
\end{document}
